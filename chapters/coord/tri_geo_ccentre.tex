%%
%	Refer to \figref{fig:tri_ccentre}.
%\subsection{Perpendicular Bisectors}
%\renewcommand{\theequation}{\theenumi}
\begin{enumerate}[label=\thesection.\arabic*.,ref=\thesection.\theenumi]
\numberwithin{equation}{enumi}
%
\item In 
	\label{prob:tri-ccentre-def}
	\figref{fig:tri-perp-bis}, 
\begin{align}
OB = OC=R, 	BD = DC.
\end{align}
Show that $OD \perp BC$.
%
\begin{figure}[!ht]
	\begin{center}
		
		\resizebox{\columnwidth}{!}{%Code by GVV Sharma
%July 7, 2023
%released under GNU GPL
%The perpendicular bisector

\begin{tikzpicture}
[scale=2,>=stealth,point/.style={draw,circle,fill = black,inner sep=0.5pt},]

%Triangle sides
\def\a{5}
\def\b{6}
\def\c{4}
 
%Coordinates of A
%\def\p{{\a^2+\c^2-\b^2}/{(2*\a)}}
\def\p{0.5}
\def\q{{sqrt(\c^2-\p^2)}}

%Labeling points
%\node (A) at (\p,\q)[point,label=above right:$A$] {};
\node (B) at (0, 0)[point,label=below left:$B$] {};
\node (C) at (\a, 0)[point,label=below right:$C$] {};
%Mid point
\node (D) at ($(B)!0.5!(C)$)[point,label=below:$D$] {};

%Circumcentre

\node (O) at (2.5,1.70084013)[point,label=above right:$O$] {};

%Drawing triangle OBC
%\draw (A) -- node[left] {$\textrm{c}$} (B) -- node[below] {$\textrm{a}$} (C) -- node[above,yshift=2mm] {$\textrm{b}$} (A);
%Drawing OA, OB, OC
%\draw (O) -- node[left] {$\textrm{R}$} (A);
\draw (O) -- node[below] {${R}$} (B);
\draw (O) -- node[below] {${R}$} (C);
\draw (B) -- (C);
%\draw (B) -- node[below] {${a}$} (C);
\draw (O) --   (D);

\tkzMarkAngle[fill=blue!50,size=.3](C,B,O)
\tkzMarkAngle[fill=blue!50,size=.3](O,C,B)


%\tkzMarkAngle[fill=red!10](O,A,C)
%\tkzMarkAngle[fill=red!10](A,C,O)

\iffalse
\tkzMarkAngle[fill=orange!50,size=.3](B,A,O)
\tkzMarkAngle[fill=orange!50,size=.3](O,B,A)

\tkzLabelAngle[pos=0.5](O,C,B){$\theta_1$}
\tkzLabelAngle[pos=0.5](O,B,C){$\theta_1$}
\tkzLabelAngle[pos=0.5](O,A,B){$\theta_2$}
\tkzLabelAngle[pos=0.5](O,B,A){$\theta_2$}
\tkzLabelAngle[pos=1.5](O,A,C){$\theta_3$}
\tkzLabelAngle[pos=1.5](O,C,A){$\theta_3$}
\fi

\end{tikzpicture}
}
	\end{center}
	\caption{Perpendicular bisector.}
	\label{fig:tri-perp-bis}	
%github/geometry/figs/
\end{figure}
\\
\solution 
\begin{align}
	\norm{\vec{O}-\vec{C}} &=
\norm{\vec{O}-\vec{B}} =R
\\
	\implies \norm{\vec{O}-\vec{C}}^2 &=
\norm{\vec{O}-\vec{B}}^2 
\end{align}
which can be expressed as 
\begin{align}
%  \label{eq:norm2d_dist}
	\brak{\vec{O}-\vec{C}}^{\top} \brak{\vec{O}-\vec{C}}&=
	\brak{\vec{O}-\vec{B}}^{\top} 
\brak{\vec{O}-\vec{B}}
\\
\norm{\vec{O}}^2-2{\vec{O}}^{\top}\vec{C} + \norm{\vec{C}}^2
	&= \norm{\vec{O}}^2-2{\vec{O}}^{\top}\vec{B} + \norm{\vec{B}}^2
	\\
	\implies 
	  \brak{\vec{B}-\vec{C}}^{\top}{\vec{O}} 
	  &=  \frac{\norm{\vec{B}}^2 - \norm{\vec{C}}^2}{2}
\end{align}
which can be simplified to obtain
  \begin{align}
	  \brak{\vec{B}-\vec{C}}^{\top}\cbrak{{\vec{O}}- 
	    \brak{\frac{{\vec{B}} + {\vec{C}}}{2}}}=0
  \label{eq:norm2d_equidist}
  \\
	  \text{or, }
	  \brak{\vec{B}-\vec{C}}^{\top}\cbrak{{\vec{O}}- \vec{D}}=0
  \end{align}
  which proves the give result using 
	  \eqref{eq:section_formula}
	  and 
  \eqref{eq:dot2d-orth}.
  \item The equation  of the circle in 
	\figref{fig:tri_ccircle-ang},
\begin{figure}[!ht]
	\begin{center}
		\resizebox{\columnwidth}{!}{%Code by GVV Sharma
%July 7, 2023
%released under GNU GPL
%The circumcircle

\begin{tikzpicture}
[scale=2,>=stealth,point/.style={draw,circle,fill = black,inner sep=0.5pt},]

%Triangle sides
\def\a{5}
\def\b{6}
\def\c{4}
\def\R{3.023715784073818}
 
%Coordinates of A
%\def\p{{\a^2+\c^2-\b^2}/{(2*\a)}}
\def\p{0.5}
\def\q{{sqrt(\c^2-\p^2)}}

% Vertices
\node (A) at (\p,\q)[point,label=above right:$A$] {};
\node (B) at (0, 0)[point,label=below left:$B$] {};
\node (C) at (\a, 0)[point,label=below right:$C$] {};
\iffalse
% Mid points
\node (D) at ($(B)!0.5!(C)$)[point,label=below:$D$] {};
\node (E) at ($(C)!0.5!(A)$)[point,label=right:$E$] {};
\node (F) at ($(B)!0.5!(A)$)[point,label=left:$F$] {};
\fi

%Circumcentre

\node (O) at (2.5,1.70084013)[point,label=right:$O$] {};

%Drawing triangle ABC
%\draw (A) -- node[above left, yshift=2mm] {$\textrm{c}$} (B) -- node[below right, xshift = 2mm] {$\textrm{a}$} (C) -- node[above,yshift=2mm] {$\textrm{b}$} (A);
\draw (A) --  (B) --  (C) --  (A);
%Drawing OA, OB, OC
%\draw (O) -- node[left] {$\textrm{R}$} (A);
\draw (O) -- node[below] {$\textrm{R}$} (B);
\draw (O) -- node[below] {$\textrm{R}$} (C);
\iffalse
%Drawing OD, OE, OF
\draw (O) --  (D);
\draw (O) --  (E);
\draw (O) --  (F);
\fi


%Drawing circumcircle
\draw (O) circle (\R);

\iffalse
\tkzMarkRightAngle[fill=blue!20,size=.2](O,D,C)
\tkzMarkRightAngle[fill=blue!20,size=.2](O,E,A)
\tkzMarkRightAngle[fill=blue!20,size=.2](O,F,B)
\fi

\tkzMarkAngle[fill=orange!10](B,O,C)
\tkzMarkAngle[fill=orange!10](B,A,C)
\tkzLabelAngle[pos=0.35](B,O,C){$\theta$}
\end{tikzpicture}
}
	\end{center}
	\caption{Circumcircle of $\triangle ABC$}
	\label{fig:tri_ccircle-ang}	
\end{figure}
	is
  \begin{align}
	  \norm{\vec{x}-\vec{O}} = R
  \end{align}
  This is known as the {\em circumcircle} of $\triangle ABC$.
  \item In 
	\figref{fig:tri_cosine_formula}	
	show that 
\begin{equation}
	\cos A= 	\frac{\brak{\vec{A}-
	\vec{B}}^{\top}\brak{\vec{A}-\vec{C}}}{\norm{\vec{A}-\vec{B}}\norm{\vec{A}-\vec{C}}}
\label{eq:tri_cos_form-ccentre}
\end{equation}
\solution
From 
\eqref{eq:tri_cos_form}, using 
  \eqref{eq:norm2d_dist},
\begin{align}
\label{eq:tri_cos_form-ccentre-norm}
	\cos A&= 	\frac{\norm{\vec{A}-\vec{B}}^2+\norm{\vec{A}-\vec{C}}^2-\norm{\vec{B}-\vec{C}}^2}{2\norm{\vec{A}-\vec{B}}\norm{\vec{A}-\vec{C}}}
	\\
	&= 	\frac{\norm{\vec{A}}^2-\vec{A}^{\top}\vec{B}-\vec{A}^{\top}\vec{C}+\vec{B}^{\top}\vec{C}}{\norm{\vec{A}-\vec{B}}\norm{\vec{A}-\vec{C}}}
\end{align}
which can be expressed as 
\eqref{eq:tri_cos_form-ccentre}.
\item Any point on the circle can be expressed as 
  \begin{align}
	  \vec{x} = \vec{O} + R\myvec{\cos \theta \\ \sin \theta}, \quad 0 \in \sbrak{0, 2\pi}
\label{eq:polar-ccentre}.
  \end{align}
  \item Let
  \begin{align}
	  R = 1,\,
	  \vec{O} = \vec{0} ,\,
	  \vec{A} = \myvec{\cos \theta_1 \\ \sin \theta_1},\,
	  \vec{B} = \myvec{\cos \theta_2 \\ \sin \theta_2},\,
  \end{align}
Show that 
  \begin{align}
	  \norm{\vec{A}-\vec{B}} = 
	   2 \sin \brak{\frac{\theta_1-\theta_2}{2}}
\label{eq:norm-polar-ccentre}
  \end{align}
  \solution 
  From 
\eqref{eq:polar-ccentre}.
  \begin{align}
	  \vec{A}-\vec{B} &= 
\myvec{\cos \theta_1-\cos \theta_2 \\ \sin \theta_1-\sin \theta_2}
\\
\implies 
	  \norm{\vec{A}-\vec{B}}^2 &= 
	  \brak{\cos \theta_1-\cos \theta_2}^2 +\brak{\sin \theta_1-\sin \theta_2}^2
	  \\
	  &= 2\cbrak{1-
	  \cos \brak{\theta_1-\theta_2}} = 4 \sin^2 \brak{\frac{\theta_1-\theta_2}{2}}
  \end{align}
  yielding 
\eqref{eq:norm-polar-ccentre} from
\eqref{eq:trig-id-2A-cos}.
  \item In 
	\figref{fig:tri_ccircle-ang},
show that 
  \begin{align}
	  \theta = 2A
\label{eq:ang-subtend-ccentre}.
  \end{align}
  \solution Let 
  \begin{align}
	  \vec{C} = \myvec{\cos \theta_3 \\ \sin \theta_3}
  \end{align}
  Then, 
  substituting 
  from 
\eqref{eq:norm-polar-ccentre}
in 
\eqref{eq:tri_cos_form-ccentre-norm},
  \begin{align}
	  \cos A &= \frac{4 \sin^2 \brak{\frac{\theta_1-\theta_2}{2}} +4\sin^2 \brak{\frac{\theta_1-\theta_3}{2}}-4 \sin^2 \brak{\frac{\theta_2-\theta_3}{2}}}{8 \sin \brak{\frac{\theta_1-\theta_2}{2}} \sin \brak{\frac{\theta_1-\theta_3}{2}}}
	  \\
	   &= \frac{2 \sin^2 \brak{\frac{\theta_1-\theta_2}{2}} +\cos \brak{{\theta_2-\theta_3}}- \cos \brak{{\theta_1-\theta_3}}}{4 \sin \brak{\frac{\theta_1-\theta_2}{2}} \sin \brak{\frac{\theta_1-\theta_3}{2}}}
  \end{align}
  from 
\eqref{eq:trig-id-2A-cos}. $\therefore$ from 
\eqref{eq:trig_id_sum_diff4},
  \begin{align}
	   \cos A &= \frac{2 \sin^2 \brak{\frac{\theta_1-\theta_2}{2}} +2\sin \brak{\frac{\theta_1-\theta_2}{2}}\sin \brak{\frac{\theta_1+\theta_2}{2}-\theta_3}}{4 \sin \brak{\frac{\theta_1-\theta_2}{2}} \sin \brak{\frac{\theta_1-\theta_3}{2}}}
	  \\
	   &= \frac{ \sin \brak{\frac{\theta_1-\theta_2}{2}} +\sin \brak{\frac{\theta_1+\theta_2}{2}-\theta_3}}{ 2\sin\brak{\frac{\theta_1-\theta_3}{2}}}
  \end{align}
  From 
\eqref{eq:trig_id_sum_diff1}, the above equation can be expressed as
  \begin{align}
\cos A	   &= \frac{ 2\sin \brak{\frac{\theta_1-\theta_3}{2}} \cos\brak{\frac{\theta_2-\theta_3}{2}}}{ 2\sin\brak{\frac{\theta_1-\theta_3}{2}}} = \cos\brak{\frac{\theta_2-\theta_3}{2}}
\label{eq:tri_ccentre_subtend-temp}
	   \\
	   \implies 2A &= \theta_2-\theta_3
\label{eq:tri_ccentre_subtend}
  \end{align}
  Similarly, 
  \begin{align}
	  \cos \theta = \frac{1 + 1 - 4\sin^2\brak{\frac{\theta_2-\theta_3}{2}}}{2} = \cos\brak{{\theta_2-\theta_3}}= \cos 2A
  \end{align}
\end{enumerate}
  \iffalse
%
\item  Show that $\angle BOC = 2\angle A$.
\label{prob:tri_ccentre_subtend}
%
\\
\solution In Fig. \ref{fig:tri_ccentre}, 
%
\begin{align}
%
\label{eq:tri_ccentre_A23}
A &= \theta_2+\theta_3
\\
B &= \theta_1+\theta_2
\\
C &= \theta_3+\theta_1
\\
\implies 2\brak{\theta_1+\theta_2+\theta_3} &= A+B+C =180\degree
\\
\implies \theta_1+\theta_2+\theta_3 &= 90\degree
\label{eq:tri_ccentre_sum_123}
\end{align}
%
From \eqref{eq:tri_ccentre_A23} and \eqref{eq:tri_ccentre_sum_123},
%
\begin{align}
%
\label{eq:tri_ccentre_A1}
A &= 90\degree - \theta_1
\end{align}
%
Also, in $\triangle OBC$, all angles add up to $180\degree$.  Hence, 
%
\begin{align}
\angle BOC + 2\theta_1 &= 180\degree
\\
\implies \angle BOC &= 180\degree - 2\theta_1 = 2 \brak{90\degree - \theta_1}
%\\
%&
= 2\angle A
\end{align}
%
upon substituting from \eqref{eq:tri_ccentre_A1}.
%
\item Let $\vec{D}$ be the mid point of $BC$.  Show that $OD \perp BC$.
\label{prob:tri_perp_bisect}
%
\\
\solution From 
	\eqref{prob:tri-ccentre-def},
%
\begin{align}
	\norm{\vec{O}-\vec{C}} = 
	\norm{\vec{O}-\vec{B}}  = R
	\\
	\implies 
	\norm{\vec{O}-\vec{C}}^2 = 
	\norm{\vec{O}-\vec{B}}^2
\brak{\vec{B}-\vec{C}}^T\vec{O} &=   \frac{\norm{\vec{B}}^2- \norm{\vec{C}}^2}{2}
\\
\implies \brak{\vec{B}-\vec{C}}^T\vec{O} &=   \frac{1}{2}\brak{\vec{B}- \vec{C}}^T\brak{\vec{B}+ \vec{C}}
\\
\implies \brak{\vec{B}-\vec{C}}^T&\brak{\vec{O} - \frac{\vec{B}+\vec{C}}{2}} = 0
\\
\text{or, } \brak{\vec{B}-\vec{C}}^T&\brak{\vec{O} - \vec{D}} = 0
\end{align}
%
$\because \vec{D} = \frac{\vec{B}+\vec{C}}{2}$ is the mid point of $BC$.  From \eqref{eq:tri_baudh_orth} we then conclude that $OD \perp BC$.
%
\item Perpendicular bisectors of a triangle meet at the circumcentre.
%
\item In the isosceles $\triangle OBC$, if $BD = DC$, $OD \perp BC$.
\label{them:isos_pb}

\item Find a point $\vec{O}$ that is equidistant from the vertices of $\triangle ABC$ for $a = 5, b = 6, c = 4$.
%
\solution Let $\vec{O}$ be the desired point.  Then,
\begin{align}
\label{eq:tri_ccentre_def}
\norm{\vec{A}-\vec{O}} = \norm{\vec{B}-\vec{O}} = 
\norm{\vec{C}-\vec{O}} = R
%\\
%\implies \norm{\vec{x}-\vec{O}}^2 &=\brak{\vec{x}-\vec{O}}^T\brak{\vec{x}-\vec{O}} = R^2
\end{align}
From \eqref{eq:tri_ccentre_def},
\begin{align}
\label{eq:circle_const_AB}
\norm{\vec{A}-\vec{O}}^2 - \norm{\vec{B}-\vec{O}}^2  = 0
\end{align}
\begin{multline}
\implies \brak{\vec{A}-\vec{O}}^T\brak{\vec{A}-\vec{O}} 
\\
- \brak{\vec{B}-\vec{O}}^T\brak{\vec{B}-\vec{O}} = 0
\end{multline}
%
which can be simplified as
\begin{align}
\label{eq:circle_const_chord_ab}
\brak{\vec{A}-\vec{B}}^T\vec{O} =   \frac{\norm{\vec{A}}^2- \norm{\vec{B}}^2}{2}
\end{align}
Similarly,
\begin{align}
\label{eq:circle_const_chord_bc}
\brak{\vec{B}-\vec{C}}^T\vec{O} =   \frac{\norm{\vec{B}}^2- \norm{\vec{C}}^2}{2}
\end{align}
%
\eqref{eq:circle_const_chord_ab} and \eqref{eq:circle_const_chord_ab}, can be combined to form the matrix equation 
%
\begin{align}
\vec{N}^T\vec{O} &= \vec{c}
\\
\implies \vec{O} &= \vec{N}^{-T} \vec{c}
\label{eq:circle_const_chord_mat}
\end{align}
%
where 
%
\begin{align}
%\label{eq:circle_const_chord_mat}
\vec{N} &= \myvec{\vec{A}-\vec{B} & \vec{B}-\vec{C}}
\\
\vec{c} &= \frac{1}{2}\myvec{\norm{\vec{A}}^2- \norm{\vec{B}}^2 \\ \norm{\vec{B}}^2- \norm{\vec{C}}^2}
\end{align}
%
$\vec{O}$ can be computed using 
%
the python code below
%
\begin{lstlisting}
codes/circle/tri_ccentre.py
\end{lstlisting}
%
and the equivalent latex-tikz code to draw Fig. \ref{fig:tri_ccentre} is
%
\begin{lstlisting}
figs/triangle/tri_ccentre.tex
\end{lstlisting}
%
\fi
  \iffalse
  \begin{align}
\brak{\vec{A}-
	  \vec{B}}^{\top}\brak{\vec{A}-\vec{C}} = \brak{\cos \theta_1-\cos \theta_2} \brak{\cos \theta_1-\cos \theta_3}+\brak{\sin \theta_1-\sin \theta_2}
  \brak{\sin \theta_1-\sin \theta_3}
  \end{align}
  \fi
