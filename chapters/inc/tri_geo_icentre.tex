%%
%\subsection{Perpendicular Bisectors}
\renewcommand{\theequation}{\theenumi}
\begin{enumerate}[label=\arabic*.,ref=\thesubsection.\theenumi]
\numberwithin{equation}{enumi}

\item Find a point $\vec{O}$ that is equidistant from the sides of $\triangle ABC$ for $a = 5, b = 6, c = 4$. Here, distance means the perpendicular distanace.
%
\solution Let $\vec{I}$ be the desired point and  $\vec{D}, \vec{E}, \vec{F}$ are on  $BC, CA, AB$ such that $ID \perp BC, IE \perp CA, IF \perp AB$ and $ID = IE = IF = r$, then, applying \eqref{eq:tri_baudh} in $\triangle s$ $IDB$ and $IEB$,
\begin{align}
\label{eq:tri_icentre_baudhd}
\begin{split}
IB^2 &= ID^2+BD^2 = r^2 + BD^2 
\\
IB^2 &= IF^2+BF^2 = r^2 + BF^2
\end{split}
\end{align}
From the above, it is obvious that $BD = BF$. Similarly, $AE = AF, CD = CF$.  Denoting these lengths as $x, y$ and $z$, as shown in Fig. \ref{fig:tri_icentre},
%
\begin{align}
x + y = a
y+z = b
x + z = c
\end{align}
%
which can be expressed as the matrix equation
%
\begin{align}
\myvec{
1 & 1 & 0
\\
0 & 1 & 1
\\
1 & 0 & 1
}
\myvec{x \\ y \\ z}
=
\myvec{a\\b\\c}
\label{eq:tri_icentre_mat}
\end{align}
%
Section formula can be used to compute 
\begin{align}
\vec{D} = \frac{x\vec{C}+y\vec{B}}{x+y}
\\
\vec{E} = \frac{y\vec{A}+z\vec{C}}{y+z}
\\
\vec{F} = \frac{z\vec{B}+x\vec{A}}{z+x}
\end{align}
%
Note that $\vec{I}$ is the circumcentre of $\triangle DEF$.  Thus, \eqref{eq:circle_const_chord_mat} can be used to compute $\vec{I}$.  
%
This is done by the python code below
%
\begin{lstlisting}
codes/circle/tri_icentre.py
\end{lstlisting}
%
and the equivalent latex-tikz code to draw Fig. \ref{fig:tri_icentre} is
%
\begin{lstlisting}
figs/circle/tri_icentre.tex
\end{lstlisting}
%
%
\item $r$ is known as the {\em inradius} of $\triangle ABC$.  Find $r$ for  the given values of $a,b,c$.
\\
\solution From Fig. \ref{fig:tri_icentre}
%
\begin{align}
\because ar\brak{ABC} &= ar\brak{IBC}+ar\brak{ICA}+ar\brak{IAB}
\\
&=\frac{1}{2}ra+\frac{1}{2}rb+\frac{1}{2}rc = \frac{a+b+c}{2}r,
\\
r &= \frac{\sqrt{s\brak{s-a}\brak{s-b}\brak{s-c}}}{s}
\end{align}
%
using Hero's formula.
%
The following python code computes the {\em inradius}
%
\begin{lstlisting}
codes/circle/tri_iradius.py
\end{lstlisting}

\end{enumerate}

